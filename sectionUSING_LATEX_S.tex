\section*{USING LATEX}
Some examples of commonly used \LaTeX{}  commands and features are listed below, to help you get started.


\subsection*{Sections}


Use section and subsection commands to organize your document. \LaTeX{} handles all the formatting and numbering automatically. Use ref and label commands for cross-references.


\subsection*{Tables}


Use the table and tabledata commands for basic tables --- see Table~\ref{tab:widgets}, for example.
\begin{table}
\hrule \vspace{0.1cm}
\caption{\label{tab:widgets}An example of a simple table with caption.}
\centering
\begin{tabledata}{llr} 
\header First name & Last Name & Grade \\ 
\row John & Doe & $7.5$ \\ 
\row Richard & Miles & $2$ \\ 
\end{tabledata}
\end{table}


\subsection*{Figures}
You can upload a figure (JPEG, PNG or PDF) using the files menu. To include it in your document, use the includegraphics command (see the example below in the source code).


Figure legends should briefly describe the key messages of the figure such that the figure can stand alone from the main text. However, all figures should also be discussed in the article text. Each legend should have a concise title of no more than 15 words. The legend itself should be succinct, while still explaining all symbols and abbreviations. Avoid lengthy descriptions of methods.


For any figures reproduced from another publication (as long as appropriate permission has been obtained from the copyright holder —see under the heading 'Submission'), please include a line in the legend to state that: 'This figure has been reproduced with kind permission from [include original publication citation]'.
